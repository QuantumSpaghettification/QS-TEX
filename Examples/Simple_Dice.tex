\documentclass[a4paper,12pt]{article}
\usepackage[a4paper, margin=0.5in]{geometry}
\usepackage{color}
\usepackage{listings}
\usepackage{graphicx}
\title{A simple Dice Roll:\\An example of QS-TEX in R}
\begin{document}
\maketitle
<script=R:action={!!sid=1!!}>
data=c(2,4,4,6,2,6,1,6,1,3)
output('')
</script>
We roll a dice <script=R:action={!!sid=1!!}>output(length(data))</script> times and get the values:
\[<script=R:action={!!sid=1!!}>
string=""
for(x in data){string=paste(string,x,sep=',')}
output(substr(string,2,nchar(string)))
</script>\]
These have a mean of <script=R:action={!!sid=1!!}>output(mean(data))</script> and a standard deviation of <script=R:action={!!sid=1!!}>output(sd(data))</script>. This gives us a boxplot of the form:
\begin{center}
<script=R:action={!!sid=1!!}>boxplot(data, xlab="", ylab="Value", main="Dice Rolls", col="red");output(imagedisp('Rplots.pdf'))</script> \end{center}

\end{document}