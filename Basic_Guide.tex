\documentclass[a4paper,12pt,twocolum]{article}
\usepackage[a4paper, margin=0.5in]{geometry}
\usepackage{color}
\usepackage{listings}
\title{QS-TEX: Basic guide}
\begin{document}
\maketitle
\section{Syntax}
The basic syntax is:
\begin{lstlisting}
<!script=lang:action={}>
....code...
output(TeX_String_To_Print)
</!script>
\end{lstlisting}
which can be placed anywhere in you LaTeX document (with **!script** replaced with **script**). The output() is outputted at the same place the code lies in your document. At the momentum the only language it supports is python2.7 with lang=py - there are plans to add more. For example:

<script=py:action={}>
text='Hello World'
output(scrinc('HF')+"output: "+text)
</script>

Here the \lstinline{scrinc('HF')} is a command to inculde the script along with the header and footer (\lstinline{<script...>} ... \lstinline{</script>}).
\section{Script Continued}
Sometimes you may want a script in one part of your document to continue in another. To do this you can use the action \lstinline{!!sid=some_id!!} e.g.

<script=py:action={!!sid=1!!}>
a=1
output(scrinc('HF'))
</script>

<script=py:action={!!sid=2!!}>
a=2
output(scrinc('HF'))
</script>

<script=py:action={!!sid=1!!}>
output(scrinc('HF')+"output: "+str(a))
</script>

<script=py:action={!!sid=2!!}>
output(scrinc('HF')+"output: "+str(a))
</script>
\end{document}