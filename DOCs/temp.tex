\documentclass[a4paper,12pt,twocolum]{article}
\usepackage[a4paper, margin=0.5in]{geometry}
\usepackage{color}
\usepackage{listings}
\title{QS-TEX: Basic guide}
\begin{document}
\maketitle
\section{Syntax}
The basic syntax is:
\begin{lstlisting}
<!script=lang:action={}>
....code...
output(TeX_String_To_Print)
</!script>
\end{lstlisting}
which can be placed anywhere in you LaTeX document (with **!script** replaced with **script**). The output() is outputted at the same place the code lies in your document.  For example:


      \definecolor{codegreen}{rgb}{0,0.6,0}
      \definecolor{codegray}{rgb}{0.5,0.5,0.5}
      \definecolor{codepurple}{rgb}{0.58,0,0.82}
      \definecolor{backcolour}{rgb}{0.95,0.95,0.92}
      \lstdefinestyle{mystyle0}{
      backgroundcolor=\color{white},   
      commentstyle=\color{codegreen},
      keywordstyle=\color{magenta},
      linewidth=5in,
      numberstyle=\tiny\color{codegray},
      stringstyle=\color{codepurple},
      basicstyle=\footnotesize,
      breakatwhitespace=false,         
      breaklines=true,                 
      captionpos=b,                    
      keepspaces=true,                 
      numbers=left,                    
      numbersep=5pt,                  
      showspaces=false,                
      showstringspaces=false,
      showtabs=false,                  
      tabsize=2
      }
      %.................................................
\begin{lstlisting}[language=python,style=mystyle0]
<script=python2.7:action={}>
text='Hello World'
output(scrinc('HF')+"output: "+text)
</script>\end{lstlisting}
          output: Hello World

Here the \lstinline{scrinc('HF')} is a command to inculde the script along with the header and footer (\lstinline{<script...>} ... \lstinline{</script>}). The currently allowed languages are
\\
\begin{center}
\begin{tabular}{|p{1in}|p{1in}|}
\hline
Lanugage & lang \\ \hline
python 2.7 & python2.7\\ \hline
R&R\\ \hline
\end{tabular}
\end{center}
\section{Script Continued}
Sometimes you may want a script in one part of your document to continue in another. To do this you can use the action \lstinline{!!sid=some_id!!} e.g.


      \definecolor{codegreen}{rgb}{0,0.6,0}
      \definecolor{codegray}{rgb}{0.5,0.5,0.5}
      \definecolor{codepurple}{rgb}{0.58,0,0.82}
      \definecolor{backcolour}{rgb}{0.95,0.95,0.92}
      \lstdefinestyle{mystyle1}{
      backgroundcolor=\color{white},   
      commentstyle=\color{codegreen},
      keywordstyle=\color{magenta},
      linewidth=5in,
      numberstyle=\tiny\color{codegray},
      stringstyle=\color{codepurple},
      basicstyle=\footnotesize,
      breakatwhitespace=false,         
      breaklines=true,                 
      captionpos=b,                    
      keepspaces=true,                 
      numbers=left,                    
      numbersep=5pt,                  
      showspaces=false,                
      showstringspaces=false,
      showtabs=false,                  
      tabsize=2
      }
      %.................................................
\begin{lstlisting}[language=python,style=mystyle1]
<script=python2.7:action={!!sid=1!!}>
a=1
output(scrinc('HF'))
</script>\end{lstlisting}
          


      \definecolor{codegreen}{rgb}{0,0.6,0}
      \definecolor{codegray}{rgb}{0.5,0.5,0.5}
      \definecolor{codepurple}{rgb}{0.58,0,0.82}
      \definecolor{backcolour}{rgb}{0.95,0.95,0.92}
      \lstdefinestyle{mystyle2}{
      backgroundcolor=\color{white},   
      commentstyle=\color{codegreen},
      keywordstyle=\color{magenta},
      linewidth=5in,
      numberstyle=\tiny\color{codegray},
      stringstyle=\color{codepurple},
      basicstyle=\footnotesize,
      breakatwhitespace=false,         
      breaklines=true,                 
      captionpos=b,                    
      keepspaces=true,                 
      numbers=left,                    
      numbersep=5pt,                  
      showspaces=false,                
      showstringspaces=false,
      showtabs=false,                  
      tabsize=2
      }
      %.................................................
\begin{lstlisting}[language=python,style=mystyle2]
<script=python2.7:action={!!sid=2!!}>
a=2
output(scrinc('HF'))
</script>\end{lstlisting}
          


      \definecolor{codegreen}{rgb}{0,0.6,0}
      \definecolor{codegray}{rgb}{0.5,0.5,0.5}
      \definecolor{codepurple}{rgb}{0.58,0,0.82}
      \definecolor{backcolour}{rgb}{0.95,0.95,0.92}
      \lstdefinestyle{mystyle3}{
      backgroundcolor=\color{white},   
      commentstyle=\color{codegreen},
      keywordstyle=\color{magenta},
      linewidth=5in,
      numberstyle=\tiny\color{codegray},
      stringstyle=\color{codepurple},
      basicstyle=\footnotesize,
      breakatwhitespace=false,         
      breaklines=true,                 
      captionpos=b,                    
      keepspaces=true,                 
      numbers=left,                    
      numbersep=5pt,                  
      showspaces=false,                
      showstringspaces=false,
      showtabs=false,                  
      tabsize=2
      }
      %.................................................
\begin{lstlisting}[language=python,style=mystyle3]
<script=python2.7:action={!!sid=1!!}>
output(scrinc('HF')+"output: "+str(a))
</script>\end{lstlisting}
          output: 1


      \definecolor{codegreen}{rgb}{0,0.6,0}
      \definecolor{codegray}{rgb}{0.5,0.5,0.5}
      \definecolor{codepurple}{rgb}{0.58,0,0.82}
      \definecolor{backcolour}{rgb}{0.95,0.95,0.92}
      \lstdefinestyle{mystyle4}{
      backgroundcolor=\color{white},   
      commentstyle=\color{codegreen},
      keywordstyle=\color{magenta},
      linewidth=5in,
      numberstyle=\tiny\color{codegray},
      stringstyle=\color{codepurple},
      basicstyle=\footnotesize,
      breakatwhitespace=false,         
      breaklines=true,                 
      captionpos=b,                    
      keepspaces=true,                 
      numbers=left,                    
      numbersep=5pt,                  
      showspaces=false,                
      showstringspaces=false,
      showtabs=false,                  
      tabsize=2
      }
      %.................................................
\begin{lstlisting}[language=python,style=mystyle4]
<script=python2.7:action={!!sid=2!!}>
output(scrinc('HF')+"output: "+str(a))
</script>\end{lstlisting}
          output: 2
\end{document}