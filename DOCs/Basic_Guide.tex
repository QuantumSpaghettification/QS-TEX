\documentclass[a4paper,12pt,twocolum]{article}
\usepackage[a4paper, margin=0.5in]{geometry}
\usepackage{color}
\usepackage{listings}
\usepackage{hyperref}
\title{QS-TEX: Basic guide}
\begin{document}
\maketitle
\section{What QS-TEX does?}
QS-TEX is a program designed to allow the implication of code within a \LaTeX  script and thus allowing scripts and documents to be kept together. 
\section{Getting Started}
To get started you need to download the following documents from:
\begin{center} \url{https://github.com/QuantumSpaghettification/QS-TEX} \end{center}
Into a directory named e.g. QS-TEX:
\begin{itemize}
\item FUNC directory
\item LANG directory
\item editor.py
\end{itemize}
Into the directory of your \LaTeX  document:
\begin{itemize}
\item QS.config
\end{itemize}
into this you must change the path to the path of your QS-TEX directory and have the option to change the sytle of listings (use `!bs!' for back slashes in this file).\\
The basic syntax for including script into your \LaTeX  program is:
<script=plain:action={!!output=scrinc()!!}>
<(!)script=lang:action={}>
your script here
</(!)script>
</script>
without the `(!)' and lang replaced by the language of your choice (see below). To compile your program you need to run (on Linux Mint\footnote{Other OS's may vary.}):
<script=plain:action={!!output=scrinc()!!}>
python2.7 <path-to-QS-TEX>/editor.py <file-Name>.tex
</script>
This will create a number of files inluding temp.tex which you should run your standard \LaTeX compiling program on.
\section{Languages}
\subsection{Languages Supported}
So far the following languages are supported.\\
\begin{tabular}{|p{1.5in}|p{1.5in}|}
\hline 
\textbf{Language} & \textbf{lang}\\
\hline
Plain Text & plain\\ \hline
Python v2.7 & python2.7\\ \hline
R & R\\ \hline
\end{tabular}
\\
at the moment the number of langauges are limited but making new ones are easy...
\section{Adding a New Language}
To create a new language you need to:
\begin{itemize}
\item add a `<name-of-language>.lang' file to the LANG directory.
\item add a `<name-of-language>' subdirectory o the FUNC directory.
\end{itemize}
The `.lang' file must take the form:
<script=plain:action={!!output=scrinc()!!}>
!!!<name-of-varible>={<Quantity>}!!!
</script>
and must include the following varibles:
\begin{itemize}
\item file\textunderscore ext: The file extention of your language.
\item lstlisting: What listing sytle you want to use (this can be left empty).
\item back\textunderscore slash\textunderscore escape: Things that need to be escaped in listings.
\item run\textunderscore command: The command line command to run your script language.
\item func\textunderscore output: A function named `output' written in the language under consideration which outputs a string to a file called 'outputfc\textunderscore !!i!!.txt` (the !!i!! is replaced by the editor.py script).
\item func.preamble: What needs to be placed at the begining of each script in this language.
\item func.scrinc: A function which outputs a string `!!norm!!' or `!!HF!!' depending on arugment and replaced `!bs!' in this string with backslashes.
\end{itemize}
\section{Actions}
Actions are included in the main header e.g.:
<script=plain:action={!!output=scrinc()!!}>
<(!)script=python2.7:action={!!<name-of-action1>!!!!<name-of-action2>!!}>
script
</(!)script>
</script>
again without the (!)'s included. Currently the actions avaible are:
\begin{itemize}
\item sid=...: scripts with the same sid (script id) are run together and sequentially, e.g. if you define a varible in one you can use it in the other etc.
\item LinNoId=...: scripts with teh same LinNoId (line number id) have their linenumbers carried on sequentially in listings.
\item output=...: Normally every script must have an output() in it (this can be left empty). The exception is when you give an output action in which case the string `...' is given as the output. This can include the scrinc() function which outputs the script.
\end{itemize}
\section{Examples}
As well as this whole document been created with the QS-TEX program; here are some basic examples of working scripts (mainly in python2.7 - sorry):
<script=python2.7:action={}>
a=1
output(scrinc('HF')+' Output:'+str(a))
</script>
\\
<script=python2.7:action={!!sid=1!!!!LinNoId=1!!}>
def adder(a):
  return(a+1)
output(scrinc('HF'))
</script>
The above function will be usuable wih any script with the same sid e.g.
<script=python2.7:action={!!sid=1!!!!LinNoId=1!!}>
output(scrinc('HF')+' Output:'+str(adder(10)))
</script>
\\
<script=python2.7:action={!!output=scrinc('HF')!!}>
x=1
</script>

\end{document}