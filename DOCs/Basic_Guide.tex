\documentclass[a4paper,12pt,twocolum]{article}
\usepackage[a4paper, margin=0.5in]{geometry}
\usepackage{color}
\usepackage{listings}
\title{QS-TEX: Basic guide}
\begin{document}
\maketitle
\section{Syntax}
The basic syntax is:
\begin{lstlisting}
<!script=lang:action={}>
....code...
output(TeX_String_To_Print)
</!script>
\end{lstlisting}
which can be placed anywhere in you LaTeX document (with **!script** replaced with **script**). The output() is outputted at the same place the code lies in your document.  For example:

<script=python2.7:action={}>
text='Hello World'
output(scrinc('HF')+"output: "+text)
</script>

Here the \lstinline{scrinc('HF')} is a command to inculde the script along with the header and footer (\lstinline{<script...>} ... \lstinline{</script>}). The currently allowed languages are
\\
\begin{center}
\begin{tabular}{|p{1in}|p{1in}|}
\hline
\textbf{Lanugage} & \textbf{lang} \\ \hline
python 2.7 & python2.7\\ \hline
R&R\\ \hline
\end{tabular}
\end{center}
\section{Script Continuation}
Sometimes you may want a script in one part of your document to continue in another. To do this you can use the action \lstinline{!!sid=some_id!!} e.g.

<script=python2.7:action={!!sid=1!!}>
a=1
output(scrinc('HF'))
</script>

<script=python2.7:action={!!sid=2!!}>
a=2
output(scrinc('HF'))
</script>

<script=python2.7:action={!!sid=1!!}>
output(scrinc('HF')+"output: "+str(a))
</script>

<script=python2.7:action={!!sid=2!!}>
output(scrinc('HF')+"output: "+str(a))
</script>
\end{document}